\documentclass[a4paper,12pt]{article}
\usepackage[utf8]{inputenc}
\usepackage[T1]{fontenc}
\usepackage{amsmath, amssymb}
\usepackage{geometry}
\geometry{margin=2.5cm}
\usepackage{graphicx}
\usepackage{wrapfig}
\usepackage{amsfonts}
\usepackage{tikz}

\title{Transformations du plan, frises et pavages}

\begin{document}

    \maketitle

    \section*{Dans le programme de :}
    \textbf{Troisième, seconde, première « Mathématiques Scientifique » et terminale STD2A}

    \subsection*{Prérequis :}
    géométrie vectorielle, barycentres

    \section{Transformations}
    \subsection{Définition}
    On appelle \textit{transformation du plan} (ou de l’espace) toute fonction bijective du
    plan (ou de l’espace), c’est-à-dire que tout point du plan (ou de l’espace) possède un et un seul
    antécédent par cette fonction. \\

    \subsection{Symètries}
    \subsubsection{Symétrie axiale}
    \textbf{Définition :} Soit $d$ une droite du plan. La symétrie axiale de centre $d$ est la transformation du plan qui à tout point $M$ du plan associe le point $M'$ tel que $d$ soit la médiatrice du segment $[MM']$. \\
    %dessin d'un exemple de symètrie axiale :
    \begin{center}
        \begin{tikzpicture}
            \draw[->] (-3,0) -- (3,0);
            \draw[->] (0,-3) -- (0,3);
            \draw (-2,-2) -- (2,2);
            \draw (-2,2) -- (2,-2);
            \draw (-2,-2) node[below left]{$M$};
            \draw (2,2) node[above right]{$M'$};
        \end{tikzpicture}
    \end{center}

\end{document}