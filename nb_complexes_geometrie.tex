\documentclass[a4paper,12pt]{article}
\usepackage[utf8]{inputenc}
\usepackage[T1]{fontenc}
\usepackage{amsmath, amssymb}
\usepackage{geometry}
\geometry{margin=2.5cm}
\usepackage{graphicx}
\usepackage{wrapfig}
\usepackage{amsfonts}
\usepackage{tikz}

\title{Utilisation des nombres complexes en Géométrie}

\begin{document}

    \maketitle

    \section*{Dans le programme de :}
    \textbf{Terminale Maths Expertes et Terminales STI2D}

    \subsection*{Prérequis :}
    Construction de l’ensemble $\mathbb{C}$ des nombres complexes, forme algébrique (opérations, propriétés, conjugué), forme trigonométrique (module, argument), suites numériques, transformations géométriques, trigonométrie.

\section*{Partie Historique}

L'étude des équations polynomiales a longtemps été au cœur de l'algèbre. Des mathématiciens comme Tartaglia, Cardan, Bombelli, Descartes et Girard ont introduit les quantités complexes sous forme symbolique.

Plus tard, Gauss, Argand et Mourey ont établi un lien entre les nombres complexes et la géométrie, notamment en essayant de formaliser la notion de vecteurs. Ce lien a permis des avancées comme le calcul de $\cos(\pi/5)$, en relation avec la construction du pentagone régulier dans les \emph{Éléments} d'Euclide. Klein, dans son programme d'Erlangen, a introduit une perspective géométrique à travers l'étude des similitudes directes du plan complexe.

Initialement développés pour des raisons internes aux mathématiques, les nombres complexes sont aujourd'hui des outils essentiels en physique (électricité) et en économie (cycles de croissance et de prix).



    \section*{Représentation des nombres complexes}
    \section{Forme algébrique}
    \textbf{Propriété :} Tout élément de $\mathbb{C}$ s’écrit de manière unique sous la forme $a + ib$ avec $a, b \in \mathbb{R}$.

    \begin{itemize}
        \item $a$ est appelé \textbf{partie réelle} de $z$, on note $a = \operatorname{Re}(z)$.
        \item $b$ est appelé \textbf{partie imaginaire} de $z$, on note $b = \operatorname{Im}(z)$.
    \end{itemize}

    \textbf{Remarque :}
    \begin{itemize}
        \item Si $a = 0$, on dit alors que $z$ est \textbf{imaginaire pur}.
        \item Si $b = 0$, alors $z = a \in \mathbb{R}$, $z$ est un \textbf{nombre réel}.
    \end{itemize}

    \subsection{Exemples :}

    \begin{itemize}
        \item Soit $z_1$ le nombre complexe tel que $z_1 = 5 - 4i$.
        \begin{itemize}
            \item La partie réelle de $z_1$: $\operatorname{Re}(z_1) = 5$.
            \item La partie imaginaire de $z_1$: $\operatorname{Im}(z_1) = -4$.
        \end{itemize}

        \item Soit $z_2$ le nombre complexe tel que $z_2 = 3,79i$.
        \begin{itemize}
            \item La partie réelle de $z_2$: $\operatorname{Re}(z_2) = 0$.
            \item La partie imaginaire de $z_2$: $\operatorname{Im}(z_2) = 3,79$.
            \item $z_2$ est \textbf{imaginaire pur}.
        \end{itemize}
    \end{itemize}

    \subsection{Définition :}
    Soit $z$ un nombre complexe tel que $z = a + ib$, avec $a$ et $b$ deux nombres réels. Alors, le \textbf{conjugué} de $z$, noté $\overline{z}$, est le nombre complexe défini par
    \[
        \overline{z} = a - ib.
    \]
    \subsection{Propriétés :}
    \begin{itemize}
        \item $\overline{\overline{z}} = z$.
        \item $\overline{z_1 + z_2} = \overline{z_1} + \overline{z_2}$.
        \item $\overline{z_1 z_2} = \overline{z_1} \cdot \overline{z_2}$.
        \item $\overline{\frac{z_1}{z_2}} = \frac{\overline{z_1}}{\overline{z_2}}$.
        \item $|z|^2 = z\overline{z}$.
    \end{itemize}
    \subsection{Exemple :}

    Soit $z$ le nombre complexe tel que $z = 3 - 7i$.

    \begin{itemize}
        \item Le conjugué de $z$ est $\overline{z} = 3 + 7i$.
    \end{itemize}

    \section{Forme trigonométrique}
    \begin{figure}[h]
        \centering
        \includegraphics[width=0.3\textwidth]{formealgebrique.png}
        \label{fig:formealgebrique}
    \end{figure}
    \subsection{Définition :}
    Soit $z$ un nombre complexe non nul et $M$ le point d’affixe $z$ (voir figure). On appelle \textbf{argument} de $z$ toute mesure en radians de l’angle $\widehat{(\vec{u}, \vec{OM})}$, avec $\vec{u}$ le vecteur unitaire de l’axe des réels positifs.

    On le note $\arg(z)$. L’argument est défini \textbf{à $2k\pi$ près} ($k \in \mathbb{Z}$).

    \subsection{Remarques :}
    \begin{enumerate}
        \item Si $z$ est un réel, c’est-à-dire $z = a$ :
        \begin{itemize}
            \item si $a > 0$, alors $|z| = a$ et $\arg(z) = 0$.
            \item si $a < 0$, alors $|z| = -a$ et $\arg(z) = \pi$.
        \end{itemize}
        \item Si $z$ est un imaginaire pur, c’est-à-dire $z = ib$ :
        \begin{itemize}
            \item si $b > 0$, alors $|z| = b$ et $\arg(z) = \frac{\pi}{2}$.
            \item si $b < 0$, alors $|z| = -b$ et $\arg(z) = -\frac{\pi}{2}$.
        \end{itemize}
    \end{enumerate}

    \subsection{Propriété : Module et argument de l’opposé et du conjugué}

    \begin{figure}[h]
        \centering
        \includegraphics[width=0.6\textwidth]{figure_exemple_trigo.png}
        \caption{Module et argument de l’opposé et du conjugué}
        \label{fig:figure_exemple_trigo}
    \end{figure}

    Soit $z$ un complexe non nul et $M_1$, $M_2$, $M_3$, et $M_4$ les points d’affixes respectives $z$, $\overline{z}$, $-z$ et $-\overline{z}$.

    Comme on peut le remarquer sur la figure \ref{fig:figure_exemple_trigo}, on a les propriétés suivantes :

    \[
        |z| = |\overline{z}| = |{-z}| = |{-\overline{z}}|
    \]

    \[
        \arg(\overline{z}) = -\arg(z) \; [2\pi]
    \]

    \[
        \arg(-z) = \pi + \arg(z) \; [2\pi]
    \]

    \[
        \arg(-\overline{z}) = \pi - \arg(z) \; [2\pi]
    \]

    \section{Forme exponentielle}

    \subsection{Fonction exponentielle complexe}

    Soit $f$ la fonction définie sur $\mathbb{R}$ par $f(\theta) = \cos(\theta) + i\sin(\theta)$.

    \begin{itemize}
        \item En utilisant les formules d’addition du cosinus et du sinus, on montre que, pour tous réels $\theta$ et $\theta'$ :
        \[
            f(\theta + \theta') = f(\theta) \times f(\theta').
        \] \\
        En effet, on a : \\
        \begin{align*}
            f(\theta + \theta') &= \cos(\theta + \theta') + i\sin(\theta + \theta') \\
            &= \cos(\theta)\cos(\theta') - \sin(\theta)\sin(\theta') + i(\cos(\theta)\sin(\theta') + \sin(\theta)\cos(\theta')) \\
            &= (\cos(\theta) + i\sin(\theta))(\cos(\theta') + i\sin(\theta')) \\
            &= f(\theta) \times f(\theta').
        \end{align*}
        De plus, $f(0) = 1$.
        \item Par analogie avec la fonction exponentielle dans $\mathbb{R}$, on pose $f(\theta) = e^{i\theta}$, soit :
        \[
            e^{i\theta} = \cos(\theta) + i\sin(\theta).
        \]
        \item On a :
        \[
            |e^{i\theta}| = 1 \quad \text{et} \quad \arg(e^{i\theta}) = \theta \; [2\pi].
        \]
    \end{itemize}

    \subsection{Exemple}

    \[
        e^{i\frac{\pi}{2}} = \cos\left(\frac{\pi}{2}\right) + i\sin\left(\frac{\pi}{2}\right) = i.
    \]

    \subsection{Définition : Forme exponentielle d’un nombre complexe}

    Tout nombre complexe $z$ non nul s’écrit sous la forme :
    \[
        z = r e^{i\theta} \quad \text{avec} \quad r = |z| \quad \text{et} \quad \theta = \arg(z) \; [2\pi].
    \]
    Cette écriture est appelée \textbf{forme exponentielle} de $z$.

    Réciproquement, si $z$ est un nombre complexe tel que $z = r e^{i\theta}$ avec $r > 0$, alors $r = |z|$ et $\theta = \arg(z) \; [2\pi]$.

    \subsection{Exemple}

    Soit $z = 1 + i$. On a :
    \[
        |z| = \sqrt{2} \quad \text{et} \quad \arg(z) = \frac{\pi}{4} \; [2\pi].
    \]
    Donc, la forme exponentielle de $z$ est :
    \[
        z = \sqrt{2} e^{i\frac{\pi}{4}}.
    \]
    \subsection{Exercice :}
    %exo1.png
    \begin{figure}[h]
        \centering
        \includegraphics[width=0.5\textwidth]{exo1.png}
        \caption{Exercice 1}
        \label{fig:exo1}
    \end{figure}

    \section*{Caractérisations d'ensembles de points}
    \section{Cercles, distance à un point}
    \begin{itemize}
        \item Déterminer l'ensemble des points z complexes tels que $|z - 4| = 5$.
    \end{itemize}
    C'est l'ensemble des points z tels que la distance de z à 4 est égale à 5.
    C'est donc le cercle de centre 4 et de rayon 5.

    %Illustration du cercle
    \begin{center}
        \begin{tikzpicture}
            % Axes
            \draw[->] (-2,0) -- (10,0) node[right] {$\Re(z)$};
            \draw[->] (0,-6) -- (0,6) node[above] {$\Im(z)$};

            % Cercle de centre (4,0) et rayon 5
            \draw[blue] (4,0) circle (5);

            % Centre du cercle
            \fill[red] (4,0) circle (0.1);
            \node[anchor=north] at (4,0) {\textcolor{red}{$4$}};
            \fill[green] (9,0) circle (0.1);
            \node[anchor=north] at (9,0) {\textcolor{green}{$9$}};
            \fill[green] (-1,0) circle (0.1);
            \node[anchor=north] at (-1,0) {\textcolor{green}{$-1$}};
            \fill[green] (4,5) circle (0.1);
            \node[anchor=north] at (4,5) {\textcolor{green}{$4 + 5i$}};
            \fill[green] (4,-5) circle (0.1);
            \node[anchor=north] at (4,-5) {\textcolor{green}{$4 - 5i$}};
        \end{tikzpicture}
    \end{center}

    \section{Médiatrices}
    \begin{itemize}
        \item Déterminer l'ensemble des points z complexes tels que $|z - 2| = |z - 2i|$.
    \end{itemize}
    C'est l'ensemble des points z tels que la distance de z à 2 est égale à la distance de z à $2i$.
    C'est donc la médiatrice du segment suivant :

    \begin{center}
        \begin{tikzpicture}
            %Mediatrice des points (2,0) et (0,2)

            %axes
            \draw[->] (-2,0) -- (4,0) node[right] {$\Re(z)$};
            \draw[->] (0,-2) -- (0,4) node[above] {$\Im(z)$};

            %Points
            \fill[red] (2,0) circle (0.1);
            \node[anchor=north] at (2,0) {\textcolor{red}{$2$}};
            \fill[red] (0,2) circle (0.1);
            \node[anchor=north] at (0,2) {\textcolor{red}{$2i$}};

            %Segment [2, 2i] en pointillets
            \draw[dashed] (2,0) -- (0,2);

            %Milieu de [2, 2i]
            \fill[green] (1,1) circle (0.1);

            %Perpendiculaire au segment [2, 2i] passant par son milieu
            \draw[blue] (3,3) -- (-2,-2);


        \end{tikzpicture}
    \end{center}

    \section{Formules}
    \subsection{Formule de Moivre}
    \textbf{Formule de Moivre :} Pour tous réels $r$ et $\theta$ et tout entier naturel $n$, on a : $(e^{i\theta})^n = e^{in\theta}$. \\ Que l'on peut également écrire : $(\cos(\theta) + i\sin(\theta))^n = \cos(n\theta) + i\sin(n\theta)$.
    \subsection{Formule d'Euler :} Pour tout réel $\theta$, on a : $e^{i\theta} = \cos(\theta) + i\sin(\theta)$.

    \section{Racines n-ièmes de l'unité}
    \subsection{Définition :} Les racines n-ièmes de l'unité sont les solutions de l'équation $z^n = 1$.
    \subsection{Propriété :} Les racines n-ièmes de l'unité sont les nombres complexes de la forme $e^{i\frac{2k\pi}{n}}$ avec $k \in \{0, 1, 2, \ldots, n-1\}$.
    \subsection{Exemple :} Les racines 4-ièmes de l'unité sont les nombres complexes de la forme $e^{i\frac{k\pi}{2}}$ avec $k \in \{0, 1, 2, 3\}$.

    %cercle trigo
    \begin{figure}[h]
        \centering
        \includegraphics[width=0.5\textwidth]{cercletrigo.png}
        \caption{Cercle trigonométrique}
        \label{fig:cercle_trigo}
    \end{figure}
    %exo2
    \begin{figure}[h]
        \centering
        \includegraphics[width=0.5\textwidth]{exo2.png}
    \end{figure}
\end{document}